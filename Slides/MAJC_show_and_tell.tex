\documentclass[11pt,a4paper,xcolor=dvipsnames, leqno]{beamer}

\usetheme{show_and_tell}

\title[Introduction to Mathematical
Reasoning. Induction \& Contradiction Proofs]{An Introduction to Mathematical
Reasoning with Applications to
Induction and Contradiction Proofs}
\author[M\textsuperscript{a} Asunci\'on Jim\'enez Cordero]
{{{\bf M\textsuperscript{a} Asunci\'on Jim\'enez Cordero\\
{\href{mailto:asuncionjc@us.es}{\tt asuncionjc@us.es}}}}}
\institute{Show & Tell Karumi} % Opcional
\date{}


\defbeamertemplate*{title page}{customized}[1][]
{\hspace*{-0.3cm}
	\centering		
  	\begin{beamercolorbox}[wd=1.05\textwidth, ht = 1.4cm, rounded = true, shadow = true]{my_palette}
   		 \centering \usebeamerfont{title}\smallskip \inserttitle
   		 \smallskip
     \end{beamercolorbox}
   \insertdate\par
 
}

\begin{document}
%%%%%%%%%%%%%%%%%%%%%%%%%%%%%%%%%%%%%%%%%%%%%%%%%%%%%%%%%%%%%%%%%%%%%%%%%%%%%%%%
\begingroup 
    \setbeamertemplate{headline}{}
    \begin{frame}
        \titlepage
\begin{center}
\insertauthor\par
\end{center}
\begin{center}
        \scriptsize Departamento de Estad\'istica e Investigaci\'on Operativa\\ and Instituto de Matem\'aticas de la Universidad de Sevilla,\\
        Sevilla, Spain
        \end{center}
        
\vspace*{0.3cm}
\begin{center}
\includegraphics[scale=0.35]{imus_basico_oct.png}
\end{center}
\begin{center}
\footnotesize{Show \& Tell Karumi}
\end{center}
\begin{center}
\footnotesize{February 1st 2019}
\end{center}
\end{frame}
\endgroup

%%%%%%%%%%%%%%%%%%%%%%%%%%%%%%%%%%%%%%%%%%%%%%%%%%%%%%%%%%%%%%%%%%%%%%%%%%%%%%%%
\begin{frame}
\begin{columns}
\begin{column}{0.5\textwidth}
\begin{flushright}
\href{https://github.com/asuncionjc}{\includegraphics[scale=0.12]{github_girl.png}}
\end{flushright}
\end{column}
\begin{column}{1\textwidth}
\begin{flushleft}
\small
\href{https://github.com/asuncionjc/Show_and_Tell_Mathematical_Proofs}{\tt https://github.com/asuncionjc/\\
Show\_and\_Tell\_Mathematical\_Proofs}
\end{flushleft}
\end{column}
\end{columns}
\begin{center}
\visible<2->{
\href{https://www.researchgate.net/profile/Asuncion_Jimenez-Cordero}{\includegraphics[scale=0.35]{researchgate.png}}
\hspace*{0.3cm}
\href{https://scholar.google.es/citations?user=JegcEYwAAAAJ&hl=es&oi=ao}{\includegraphics[scale=0.12]{google_scholar.png}}
}
\end{center}

\end{frame}

%%%%%%%%%%%%%%%%%%%%%%%%%%%%%%%%%%%%%%%%%%%%%%%%%%%%%%%%%%%%%%%%%%%%%%%%%%%%%%%%

\AtBeginSection[]{
\begin{without_headline}
  \begin{frame}{Outline}
    \tableofcontents[currentsection]
  \end{frame}
  \end{without_headline}
}


\begin{without_headline}
    \begin{frame}{Outline} % and our simple frame
        \tableofcontents
    \end{frame}
\end{without_headline}

%%%%%%%%%%%%%%%%%%%%%%%%%%%%%%%%%%%%%%%%%%%%%%%%%%%%%%%%%%%%%%%%%%%%%%%%%%%%%%%
\section{What is a proof?}
\begin{frame}
A \texthighlighted{theorem} is a statement that can be shown to be true.
\begin{itemize}
\item <2->Hypothesis.
\item <2->Thesis.
\end{itemize}
\begin{block}{Example}<3->
\visible<3->{\textbf{Bolzano's theorem:} If $f$ is a continuous function defined on a closed interval  $[a, b]$ such that $sign(f(a)) \neq sign(f(b))$.\\
Then there exists a point $c$ in the open interval $(a, b)$ satisfying $f(c) = 0$.}
\begin{itemize}
\item<4-> \textbf{Hypothesis:}
\begin{itemize}
\item<5-> $f:[a, b]\rightarrow \mathbb{R}$ is continuous.
\item<5-> $sign(f(a)) \neq sign(f(b))$
\end{itemize}
\item<4-> \textbf{Thesis:}
\begin{itemize}
\item<6-> $\exists c\in (a, b): f(c) = 0$
\end{itemize}
\end{itemize}
\end{block}
\end{frame}
%%%%%%%%%%%%%%%%%%%%%%%%%%%%%%%%%%%%%%%%%%%%%%%%%%%%%%%%%%%%%%%%%%%%%%%%%%%%%%%

\begin{frame}
A \texthighlighted{proof} is a chain of logical deductions that establishes the truth of a theorem.
\begin{itemize}
\item<2-> Hypothesis.
\item<2-> Axioms (\emph{Ex: A number is equal to itself, $a = a$}).
\item<2-> Previous mathematical results.
\end{itemize}
\begin{columns}
\begin{column}{0.5\textwidth}
\vspace*{-0.1cm}
\visible<3->{\begin{alertblock}{Mistakes in proofs}
\textbf{Conjecture:} 1 = 2.\\
\emph{Assume $a, b$ are two equal positive integers:}
\begin{enumerate}
\item $a = b$
\item $a^2 = ab$
\item $a^2 - b^2 = ab -b^2$
\item $(a - b)(a+ b) = b(a - b)$
\item $a + b = b$
\item $2b = b$
\item $2 = 1$
\end{enumerate}
\end{alertblock}}
\end{column}
\begin{column}{0.5\textwidth}
\visible<4->{\texthighlighted{\Large Where is the error?}}
\end{column}
\end{columns}

\end{frame}

%%%%%%%%%%%%%%%%%%%%%%%%%%%%%%%%%%%%%%%%%%%%%%%%%%%%%%%%%%%%%%%%%%%%%%%%%%%%%%%
\section{Proof by contradiction}

\begin{frame}
\begin{enumerate}
\item Assume the thesis is false.
\item Show that this assumption leads to a contradiction.
\item Hence, the thesis is true.
\end{enumerate}
\begin{alertblock}{Example}<2->
\textbf{Theorem:} $\sqrt{2}$ is irrational.
\end{alertblock}

\begin{block}{Previous concepts}<3->
\begin{itemize}
\item \textbf{Definition:} A real number $r$ is \emph{rational} if there exists integers $p$ and $q$, with no common factors and $q\neq 0$ such that $r = p/q$. A real number that is not rational is called \emph{irrational}.
\item \textbf{Theorem:} Let $a$ be an integer number such that $a^2$ is even. Then $a$ is also even.
\end{itemize}
\end{block}
\end{frame}

%%%%%%%%%%%%%%%%%%%%%%%%%%%%%%%%%%%%%%%%%%%%%%%%%%%%%%%%%%%%%%%%%%%%%%%%%%%%%%%
\begin{frame}
\begin{alertblock}{Example}
\textbf{Theorem:} $\sqrt{2}$ is irrational.
\begin{enumerate}
\item<2-> Assume $\sqrt{2}$ is rational.
\item<3-> $\sqrt{2} = \frac{p}{q}$, $p$ and $q$ without common factors.
\item<4-> $2 = \frac{p^2}{q^2}\Rightarrow p^2 = 2q^2.$
\item<5-> $p^2$ is even and, by the previous theorem, $p$ is also even.
\item<6-> $p = 2s$, for some integer $s$.
\item <7->$(2s)^2 = 2q^2 \Rightarrow 4s^2 = 2q^2 \Rightarrow 2s^2 = q^2$.
\item<8-> $q^2$ is even, and so it is $q$.
\end{enumerate}
\visible<9->{{\Large \texthighlighted{Contradiction! }}\\
\visible<10->{On the one hand, we assumed $p$ and $q$ have no common factors.\\
On the other hand, if $p$ and $q$ are even, $2$ is a common factor.\\
Therefore, \textbf{$\boldsymbol{\sqrt{2}}$ is irrational}.}}
\end{alertblock}
\end{frame}

%%%%%%%%%%%%%%%%%%%%%%%%%%%%%%%%%%%%%%%%%%%%%%%%%%%%%%%%%%%%%%%%%%%%%%%%%%%%%%%
\begin{frame}{It's your turn!}
\begin{block}{}
\textbf{Theorem:} Let $a$ be an integer number such that $a^2$ is even. Then $a$ is also even.
\end{block}
\visible<2->{
\begin{enumerate}
\item Assume $a$ is odd.
\item $a = 2k + 1$ for some integer $k$.
\item $a^2 = (2k + 1)^2 = 4k^2 + 4k + 1 = 2(2k^2 + 2k) + 1$.
\item $a^2$ is odd.
\end{enumerate}
\hspace*{0.3cm}{\large \texthighlighted{Contradiction! }} \\
\hspace*{0.3cm}Since, by hypothesis, $a^2$ is even.\\
\hspace*{0.3cm}Therefore, \textbf{$\boldsymbol{a}$ is even}.
}
\end{frame}

%%%%%%%%%%%%%%%%%%%%%%%%%%%%%%%%%%%%%%%%%%%%%%%%%%%%%%%%%%%%%%%%%%%%%%%%%%%%%%%
\begin{frame}{Another example?}
\begin{block}{}
\textbf{Theorem:}  For every natural number $a>2$ and $a$ prime, then it holds that $a$ is odd.
\end{block}
\visible<2->{
\begin{enumerate}
\item Assume $a$ is even.
\item $a = 2k$, for some $k$. Since $a>2$, then $k\neq 1$.
\item $a$ is composite.
\end{enumerate}
\hspace*{0.3cm}{\large \texthighlighted{Contradiction! }} \\
\hspace*{0.3cm}Since, by hypothesis, $a$ is prime.\\
\hspace*{0.3cm}Therefore, $\boldsymbol{a}$ \textbf{is odd}.
}
\end{frame}
%%%%%%%%%%%%%%%%%%%%%%%%%%%%%%%%%%%%%%%%%%%%%%%%%%%%%%%%%%%%%%%%%%%%%%%%%%%%%%%
\section{Proof by induction}
\begin{frame}
\begin{enumerate}
\item Base case ($n=1$).
\item Assume the result for all $k< n$ (Induction hypothesis).
\item Prove for $n$.
\end{enumerate}
\begin{alertblock}{Example}<2->
\textbf{Theorem (Gauss):} For all $n\in\mathbb{N}$:
\begin{equation*}
\sum\limits_{i = 1}^n i = \frac{n(n+1)}{2}
\end{equation*}
\begin{enumerate}
\item<3-> $(n=1) \sum\limits_{i = 1}^1 i = 1 = \frac{1(1+1)}{2}$.
\item<4-> (Induction hypothesis) The result is true for all $k<n$.
\item<5-> $\sum\limits_{i = 1}^n i = \left( \sum\limits_{i = 1}^{n-1} i \right)+ n = \frac{(n-1)(n-1 + 1)}{2}+ n = \frac{(n-1)n}{2}+ \frac{2n}{2} = \frac{n^2 -n + 2n}{2} = \frac{n^2 + n}{2} = \frac{n(n+1)}{2}$.
\end{enumerate}
\end{alertblock}
\end{frame}
%%%%%%%%%%%%%%%%%%%%%%%%%%%%%%%%%%%%%%%%%%%%%%%%%%%%%%%%%%%%%%%%%%%%%%%%%%%%%%%
\begin{frame}{Your turn!}
\begin{block}{}
\textbf{Theorem (Geometric series):} For all $n\in\mathbb{N}$ and $r\neq 1$:
\begin{equation*}
\sum\limits_{i = 0}^n r^i = \frac{1 - r^{n+1}}{1 - r}
\end{equation*}
\end{block}
\visible<2->{
\begin{enumerate}
\item $(n = 0): \sum\limits_{i = 0}^0 r^i = r^0 = 1 = \frac{1-r}{1-r}$.
\item (Induction hypothesis) The result is true for all $k<n$.
\item $\sum\limits_{i = 0}^n r^i = \sum\limits_{i = 0}^{n-1} r^i  + r^n = \frac{1 - r^n}{1 - r} + r^n = \frac{1 - r^n + r^n(1-r)}{1- r}= \frac{1 - r^n + r^n - r^{n+1}}{1-r} = \frac{1 - r^{n+1}}{1-r}$.
\end{enumerate}
}
\end{frame}


%%%%%%%%%%%%%%%%%%%%%%%%%%%%%%%%%%%%%%%%%%%%%%%%%%%%%%%%%%%%%%%%%%%%%%%%%%%%%%%
\begin{frame}{Last example}
\begin{block}{}
\textbf{Theorem:} For all $n\in\mathbb{N}$:
\begin{equation*}
11^n - 6 \text{ is divisible by } 5
\end{equation*}
\end{block}
\visible<2->{
\begin{enumerate}
\item $(n = 1): 11^1 - 6 = 5$ which is divisible by $5$.
\item (Induction hypothesis) The result is true for all $k<n$, and therefore $11^k = 5m + 6$, for some integer $m$.
\item $11^n - 6 = \left(11\cdot11^{n-1} \right) - 6 = 11\cdot (5m + 6) - 6 = 11\cdot 5m +66 - 6 = 11\cdot 5m +60 = 5\cdot (11m + 12)$, which is divisible by $5$.
\end{enumerate}
}
\end{frame}

%%%%%%%%%%%%%%%%%%%%%%%%%%%%%%%%%%%%%%%%%%%%%%%%%%%%%%%%%%%%%%%%%%%%%%%%%%%%%%%
\section{Conclusions}
\begin{frame}
\begin{itemize}
\item Some basic concepts about the mathematical results and how to prove them.
\item Proof by contradiction.
\item Proof by induction.
\end{itemize}
\end{frame}

%%%%%%%%%%%%%%%%%%%%%%%%%%%%%%%%%%%%%%%%%%%%%%%%%%%%%%%%%%%%%%%%%%%%%%%%%%%%%%%
\begingroup 
    \setbeamertemplate{headline}{}
    \begin{frame}
        \titlepage
\begin{center}
\insertauthor\par
\end{center}
\begin{center}
        \scriptsize Departamento de Estad\'istica e Investigaci\'on Operativa\\ and Instituto de Matem\'aticas de la Universidad de Sevilla,\\
        Sevilla, Spain
        \end{center}
\begin{center}
\LARGE \texthighlighted{Thank you for your attention!}
\end{center}
\begin{center}
\includegraphics[scale=0.35]{imus_basico_oct.png}
\end{center}
\begin{center}
\footnotesize{Show \& Tell Karumi}
\end{center}
\begin{center}
\footnotesize{February 1st 2019}
\end{center}
\end{frame}
\endgroup

\end{document}
